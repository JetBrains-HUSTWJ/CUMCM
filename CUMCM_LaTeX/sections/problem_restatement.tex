\section{问题重述}

\subsection{问题背景}

在现代军事对抗中,精确制导武器威胁日益增大,烟幕干扰作为重要的光电对抗技术,通过在导弹与目标间形成遮蔽云团来阻断导弹光学制导系统对目标的观测。随着无人机技术发展,利用无人机携带烟幕弹实施机动干扰成为新兴防护手段,具有机动性强、部署灵活、响应迅速等优势,能根据威胁态势动态调整干扰位置和时机,实现精确高效的防护效果。本题研究利用无人机投放烟幕干扰弹的最优策略问题,导弹采用光学制导以恒定速度直线飞向目标,无人机携带烟幕弹按预定策略投放,烟幕弹爆炸后形成球状烟幕云团遮挡导弹视线实现干扰。

\subsection{问题要求}

\textbf{问题1} 利用无人机$\text{FY1}$投放1枚烟幕干扰弹实施对$\text{M1}$的干扰,若$\text{FY1}$以$120\text{m/s}$的速度朝向假目标方
向飞行,受领任务$1.5\text{s}$后即投放1枚烟幕干扰弹,间隔$3.6\text{s}$后起爆。请给出烟幕干扰弹对$\text{M1}$的有效遮蔽时长。

\textbf{问题2} 利用无人机$\text{FY1}$投放1枚烟幕干扰弹实施对$\text{M1}$的干扰,确定$\text{FY1}$的飞行方向、飞行速度、烟幕干扰弹投放
点、烟幕干扰弹起爆点,使得遮蔽时间尽可能长。

\textbf{问题3} 利用无人机$\text{FY1}$投放3枚烟幕干扰弹,实施对$\text{M1}$的干扰。请给出烟幕干扰弹的投放策略。

\textbf{问题4} 利用$\text{FY1}$、$\text{FY2}$、$\text{FY3}$等3架无人机,各投放1枚烟幕干扰弹,实施对$\text{M1}$的干扰。请给出烟幕
干扰弹的投放策略。

\textbf{问题5} 利用5架无人机,每架无人机至多投放3枚烟幕干扰弹,实施对$\text{M1}$、$\text{M2}$、$\text{M3}$等3枚来袭导弹的干扰。请给
出烟幕干扰弹的投放策略。
