\section{附录:核心代码}

本附录包含四个问题的五个核心算法实现代码。

\subsection{第一问解法一:三维阴影投影算法}

\begin{lstlisting}[language=Matlab,caption={第一问解法一:三维阴影投影核心代码}]
% 球体和光源参数
sphere_radius = 0.2;
sphere_center = [8, 0, 4.6];
light_source = [9.5, 0, 4.5];

% 创建网格
grid_size = 300;
y_range = linspace(-5, 5, grid_size);
z_range = linspace(0, 10, grid_size);
[Y, Z] = meshgrid(y_range, z_range);
shadow_map = zeros(grid_size, grid_size);

% 计算阴影投影
for i = 1:grid_size
    for j = 1:grid_size
        point_on_plane = [0, Y(i,j), Z(i,j)];
        ray_vector = point_on_plane - light_source;
        ray_length = norm(ray_vector);
        
        if ray_length < 1e-10
            continue;
        end
        
        ray_direction = ray_vector / ray_length;
        oc = light_source - sphere_center;
        
        % 射线与球体相交检测
        a = dot(ray_direction, ray_direction);
        b = 2 * dot(oc, ray_direction);
        c = dot(oc, oc) - sphere_radius^2;
        discriminant = b^2 - 4*a*c;
        
        if discriminant < 0
            continue;
        end
        
        t1 = (-b - sqrt(discriminant)) / (2*a);
        t2 = (-b + sqrt(discriminant)) / (2*a);
        
        if (t1 > 0 && t1 < ray_length) || (t2 > 0 && t2 < ray_length)
            shadow_map(i,j) = 1;
        end
    end
end
\end{lstlisting}

\subsection{第一问解法二:点到直线距离算法}

\begin{lstlisting}[language=Matlab,caption={第一问解法二:点到直线距离核心代码}]
% 定义符号变量
syms t real;

% 导弹和目标轨迹
P1 = [18477.5931-298.5111571*t, 0, 1847.75931-29.85111571*t];
P2 = [0, 207, 0];
P3 = [0, 207, 10];
Q = [17188, 0, 1736.496-3*t];

% 计算点到直线距离
v = P2 - P1;
w = Q - P1;
cross_vw = cross(v, w);
norm_cross = norm(cross_vw);
norm_v = norm(v);
dis = norm_cross / norm_v;

% 计算第二条边的距离
v1 = P3 - P1;
cross_v1w = cross(v1, w);
norm_cross1 = norm(cross_v1w);
norm_v1 = norm(v1);
dis1 = norm_cross1 / norm_v1;

% 转换为数值函数并求解
dis_func = matlabFunction(dis);
t_range = linspace(0, 4.3535830038, 100000);
dis_values = arrayfun(dis_func, t_range);
dis_func1 = matlabFunction(dis1);
dis_values1 = arrayfun(dis_func1, t_range);

% 找到满足条件的时间范围
valid_indices = find(dis_values <= 10 & dis_values1 <= 10);
valid_t = t_range(valid_indices);
\end{lstlisting}

\subsection{第二问:单无人机多目标遗传算法优化}

\begin{lstlisting}[language=Matlab,caption={第二问核心优化代码}]
% 物理参数初始化
G = 9.8;  % 重力加速度
P_M1_INITIAL = [20000.0, 0.0, 2000.0];  % 导弹初始位置
P_FY1_INITIAL = [17800.0, 0.0, 1800.0]; % 无人机初始位置
ORIGIN = [0.0, 0.0, 0.0];               % 假目标位置

% 目标和云团参数
TARGET_RADIUS = 7.0;
TARGET_HEIGHT = 10.0;
SPEED_M1 = 300.0;
CLOUD_RADIUS = 10.0;
CLOUD_EFFECTIVE_DURATION = 20.0;
CLOUD_SINK_SPEED = 3.0;

% 遗传算法参数
POP_SIZE = 400;
MAX_GENERATIONS = 200;
CROSSOVER_RATE = 0.8;
MUTATION_RATE = 0.1;
ELITE_COUNT = 20;

% 优化变量范围 [方向角, 速度, 释放时间, 爆炸延迟]
BOUNDS = [
    0, pi;      % 飞行方向角 (rad)
    70, 140;    % 飞行速度 (m/s)
    0, 5;       % 释放时间 (s)
    0, 5        % 爆炸延迟 (s)
];

% 目标采样策略
TOTAL_POINTS = 2000;
SAMPLE_POINTS = 400;
sample_indices = round(linspace(1, TOTAL_POINTS, SAMPLE_POINTS));

% 执行多目标遗传算法优化
[best_position, best_value, covered_targets] = ga_multi(BOUNDS, POP_SIZE, ...
    MAX_GENERATIONS, CROSSOVER_RATE, MUTATION_RATE, ELITE_COUNT, ...
    P_M1_INITIAL, P_FY1_INITIAL, ORIGIN, SPEED_M1, ...
    CLOUD_SINK_SPEED, CLOUD_EFFECTIVE_DURATION, CLOUD_RADIUS, G, ...
    sample_indices, MISSILE_UNIT_TO_FAKE, use_gpu);
\end{lstlisting}

\subsection{第三问:三颗烟雾弹序列投放优化}

\begin{lstlisting}[language=Matlab,caption={第三问核心优化代码}]
% 场景参数
G = 9.8;
P_M1_INITIAL = [20000.0, 0.0, 2000.0];
P_FY1_INITIAL = [17800.0, 0.0, 1800.0];
ORIGIN = [0.0, 0.0, 0.0];

% 目标和云团参数
TARGET_CENTER_BOTTOM = [0.0, 200.0, 0.0];
TARGET_RADIUS = 7.0;
TARGET_HEIGHT = 10.0;
SPEED_M1 = 300.0;
CLOUD_RADIUS = 10.0;
CLOUD_EFFECTIVE_DURATION = 20.0;
CLOUD_SINK_SPEED = 3.0;

% 遗传算法参数
POP_SIZE = 200;
MAX_GENERATIONS = 20;

% 三颗烟雾弹的优化变量范围
BOUNDS = [
    -pi, pi;     % theta: 飞行方向角 (rad)
    70, 140;     % v: 飞行速度 (m/s)
    0, 10;       % t_release1: 第一颗释放时间 (s)
    0, 10;       % t_explode1: 第一颗爆炸延迟 (s)
    1, 5;        % x: 第二颗间隔 (s)
    0, 10;       % t_explode2: 第二颗爆炸延迟 (s)
    1, 5;        % y: 第三颗间隔 (s)
    0, 10        % t_explode3: 第三颗爆炸延迟 (s)
];

% 目标点采样
TOTAL_POINTS = 2000;
SAMPLE_POINTS = 200;
sample_indices = round(linspace(1, TOTAL_POINTS, SAMPLE_POINTS));

% 执行多目标优化
[best_position, best_value, covered_targets] = ga_multi(POP_SIZE, ...
    MAX_GENERATIONS, BOUNDS, P_M1_INITIAL, P_FY1_INITIAL, ORIGIN, ...
    SPEED_M1, CLOUD_SINK_SPEED, CLOUD_EFFECTIVE_DURATION, ...
    CLOUD_RADIUS, G, sample_indices);
\end{lstlisting}

\subsection{第四问:三架无人机协同优化}

\begin{lstlisting}[language=Matlab,caption={第四问核心优化代码}]
% 物理常量
G = 9.806;

% 初始条件:导弹和三架无人机位置
P_M1_INITIAL = [20000.0, 0.0, 2000.0];
P_FY1_INITIAL = [17800.0, 0.0, 1800.0];
P_FY2_INITIAL = [12000.0, 1400.0, 1400.0];
P_FY3_INITIAL = [6000.0, -3000.0, 700.0];
ORIGIN = [0.0, 0.0, 0.0];

% 目标参数
TARGET_CENTER_BOTTOM = [0.0, 200.0, 0.0];
TARGET_RADIUS = 7.0;
TARGET_HEIGHT = 10.0;
SPEED_M1 = 300.0;

% 云团参数
CLOUD_RADIUS = 10.0;
CLOUD_EFFECTIVE_DURATION = 20.0;
CLOUD_SINK_SPEED = 3.0;

% 遗传算法参数
POP_SIZE = 1000;
MAX_GENERATIONS = 10000;

% 三架无人机的优化变量范围
BOUNDS_UAV1 = [-pi, pi; 70, 140; 0, 7; 0, 19];
BOUNDS_UAV2 = [pi, 2*pi; 70, 140; 10, 70; 0, 17];
BOUNDS_UAV3 = [0, pi; 70, 140; 22, 70; 0, 12];
BOUNDS = [BOUNDS_UAV1; BOUNDS_UAV2; BOUNDS_UAV3];

% 目标采样策略
TOTAL_POINTS = 2000;
SAMPLE_POINTS = 40;
sample_indices = round(linspace(1, TOTAL_POINTS, SAMPLE_POINTS));

% 执行三架无人机协同优化
[best_position, best_value, covered_targets] = ga_multi(POP_SIZE, ...
    MAX_GENERATIONS, BOUNDS, P_M1_INITIAL, P_FY1_INITIAL, ...
    P_FY2_INITIAL, P_FY3_INITIAL, ORIGIN, SPEED_M1, ...
    CLOUD_SINK_SPEED, CLOUD_EFFECTIVE_DURATION, CLOUD_RADIUS, ...
    G, sample_indices);
\end{lstlisting}