\begin{abstract}

本文针对复杂电磁环境下无人机利用烟幕弹进行目标防护的策略优化问题,构建了一套完整的多层次数学模型与算法框架。

\textbf{问题一:}针对单无人机单目标烟幕干扰场景,建立了高精度的三维空间运动学模型,精确刻画导弹、无人机与烟幕弹的动态轨迹。提出了基于几何相交判定的遮蔽有效性计算方法,通过建立射线-球体相交模型和点到直线距离公式,实现了对导弹视线遮蔽状态的准确判定。设计了"粗略扫描-精确求解"的两阶段算法,高效地确定了最优的单次投弹窗口,实现了\textbf{1.39秒}的有效遮蔽时间,为后续复杂场景奠定了理论基础。

\textbf{问题二:}将单目标扩展为多目标点烟幕干扰优化问题,采用均匀网格划分策略将圆柱体表面离散化为2000个目标点。构建了以"最大化平均遮蔽时间"和"最大化目标点覆盖率"为核心的多目标优化模型,采用基于约束值的遗传算法进行求解。通过种群进化寻找最优解,平衡覆盖率和平均遮蔽时间两个目标,实现了100\%目标覆盖率和\textbf{4.80秒}平均遮蔽时间的优化效果。

\textbf{问题三:}研究多枚烟幕干扰弹的协同投放策略,优化无人机飞行方向、速度以及三枚烟幕弹的投放时间和起爆延迟等多个决策变量。建立了多云团协同遮蔽判定模型,通过逻辑或运算确定多枚烟幕云团的协同遮蔽效果。采用改进的遗传算法求解8维优化空间的多变量非线性约束优化问题,实现了连续有效遮蔽,平均干扰时间达到\textbf{6.26秒},显著提升了总体干扰效果。

\textbf{问题四:}针对多无人机协同烟幕干扰策略优化,设计了三架无人机的协同作战方案。构建了分层递进的数学模型,包括单机运动学模型、多云团时空分布模型和协同遮蔽判定模型。采用基于约束处理的改进遗传算法,通过分层约束处理机制和自适应可行性判定策略,有效解决了12维优化空间中的强约束条件下收敛困难问题,实现了\textbf{10.26秒}的协同干扰时间,展现了多机协同的显著优势。

本文的研究特色在于,将高保真的物理仿真与先进的运筹优化算法相结合,形成了一套从单体智能到群体协同的、可扩展的建模与求解体系。研究成果不仅为烟幕弹的战术应用提供了定量化的决策依据,也为解决类似的动态资源分配与路径规划问题提供了有价值的参考。

\keywords{遗传算法\quad  烟幕干扰\quad   多目标优化\quad  运筹学}
\end{abstract}